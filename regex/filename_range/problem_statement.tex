\subsection{Filename Range}
When you rename a file, cyber-dojo tries to be helpful and selects only the part of the filename it thinks you will want to change.

Your task is to write a function that is given a string filename and which returns a pair of integers specifying the start and end indexes (into the filename) of the selected range.

First, it assumes you will want to keep the extension.
Eg

   "hiker.cpp" ==> "hiker" is selected.
   "diamond.h" ==> "diamond" is selected.

Second, if the filename includes the word "tests", "test", "spec", or
"step" (case insensitive) it assumes you will want to keep those too,
together with any 'separator' characters (dot, underscore, hyphen).
Eg
\begin{verbatim}
   "HikerTest.js"  ==> "Hiker" is selected.
   "Diamond_Spec.feature" => "Diamond" is selected.
   "fizz.buzz-tests.js" => "fizz.buzz" is selected.
\end{verbatim}

Third, if the filename is in a dir/
it assumes you will also want to keep that.
Eg
\begin{verbatim}
   "test/FizzBuzz_test.exs" => "FizzBuzz" is selected.
   "src/test/Roman.spec.re" => "Roman" is selected.
\end{verbatim}

Here is a JSON data structure you can use in your tests.

\begin{verbatim}
{
  "src/Hiker_spec.re": [4,9],
  "test/hiker_test.exs": [5,10],
  "wibble/test/hiker_spec.rb": [12,17],
  "hiker_steps.rb": [0,5],
  "hiker_spec.rb": [0,5],
  "test_hiker.rb": [5,10],
  "test_hiker.py": [5,10],
  "test_hiker.sh": [5,10],
  "tests_hiker.sh": [6,11],
  "test_hiker.coffee": [5,10],
  "hiker_spec.coffee": [0,5],
  "hikerTest.chpl": [0,5],
  "hiker.tests.c": [0,5],
  "hiker_tests.c": [0,5],
  "hiker_test.c": [0,5],
  "hiker_Test.c": [0,5],
  "HikerTests.cpp": [0,5],
  "hikerTests.cpp": [0,5],
  "HikerTest.cs": [0,5],
  "HikerTest.java": [0,5],
  "DiamondTest.kt": [0,7],
  "HikerTest.php": [0,5],
  "hikerTest.js": [0,5],
  "hiker-test.js": [0,5],
  "hiker-spec.js": [0,5],
  "hiker.test.js": [0,5],
  "hiker.tests.ts": [0,5],
  "hiker_tests.erl": [0,5],
  "hiker_test.clj": [0,5],
  "fizzBuzz_test.d": [0,8],
  "hiker_test.go": [0,5],
  "hiker.tests.R": [0,5],
  "hikertests.swift": [0,5],
  "HikerSpec.groovy": [0,5],
  "hikerSpec.feature": [0,5],
  "hiker.feature": [0,5],
  "hiker.fun": [0,5],
  "hiker.t": [0,5],
  "hiker.plt": [0,5],
  "hiker": [0,5],
};
\end{verbatim}
